\documentclass[a4paper, 12pt]{article}
\usepackage{graphicx}
\usepackage{epsfig}
\usepackage{epstopdf}
\usepackage{caption}
\usepackage{float}
\graphicspath{{/Users/hyowonshin/Library/texmf/}}
\captionsetup[figure]{font=small, labelfont=small}
\begin{document}
\thispagestyle{plain}
\begin{center}
	\large
	WORKING PAPER
	
	\vspace{0.8cm}
	\Large
	\textbf{Outgroup Trust and Ethnic Voting in New Democracies: Evidence from Sub-Saharan Africa\footnote{This study has been pre-registered on the OSF website on March 26, 2021. Please refer to this link: https://osf.io/4pn6e.}}
	
	\vspace{0.4cm}
	\large
	Hyo-Won Shin\footnote{PhD candidate at the University of Illinois, Urbana-Champaign. hwshin2@illinois.edu.}

	\vspace{1cm}
	\large
	This version: May 24, 2021
	
	\vspace{1cm}
	\textbf{Abstract}
\end{center}
This study examines the relationship between out-group trust and ethnic voting across new democracies in Sub-Saharan Africa. I propose two mechanisms through, which out-group trust influences voting behavior in ethnically salient contexts. The information receptivity mechanism hypothesizes that voters with high levels of out-group trust and has greater access to information on candidates are less likely to vote for a co-ethnic candidate. The collective action mechanism proposes that individuals with high levels of out-group trust and high level of information on the voting intentions of co-ethnic and non-co-ethnic members are less likely to vote for a co-ethnic candidate. I test the relationship between out-group trust and ethnic voting using the Wave 3 Afrobarometer survey data for 10 Sub-Saharan new democracies. Results derived from a multilevel model show support for the hypothesis that individuals with high levels of out-group trust are less likely to vote for a co-ethnic candidate. Furthermore, results show mixed outcomes for the information receptivity mechanism, where individuals with high levels of information and high level of out-group trust are no more likely to vote for a co-ethnic than those with high levels of information and low level of out-group trust. Only those with a middle level of information and high levels of out-group trust are less likely to vote for a co-ethnic than those with middle level of information and low levesl of out-group trust. I find no support for the collective action mechanism.\\

\pagebreak
\section{Introduction}
For many years, scholars have argued that divisions along ethnic lines may be detrimental to the consolidation of democracy\cite{dahlPolyarchyParticipationOpposition1973, horowitzEthnicGroupsConflict1985, lijphartDemocracyPluralSocieties1977, rabushkaPoliticsPluralSocieties1972}. In societies where people identify strongly with their ethnicity, political outcomes such as voting\cite{adidaAfricanVotersFavor2015, barretoISiSePuede2007}, redistribution\cite{houleInequalityEthnicDiversity2017}, and conflict\cite{caselliTheoryEthnicConflict2013, kingDiversityViolenceRecognition2020} also tend to be divided along ethnic lines. A strong association between political outcomes and ethnic identity can be harmful to democratic consolidation as it may undermine democratic accountability, political stability, and social harmony.
\paragraph{}
Voting, a key feature of democracy, has been found to hinder democratic consolidation when done along ethnic lines\cite{houleDoesEthnicVoting2018}. Ethnic voting is harmful to democracy as it 1) reduces ex ante uncertainty of voting, 2) encourages patronage politics, and 3) pushes candidates to take extreme policy stances leading to polarization. As a result, countries that vote along ethnic lines may appear to be moving towards democracy, as voting is deemed the essence of democracy, but in actuality, they may be experiencing political patterns that in fact are preventing democratic consolidation.
\paragraph{}
If ethnic voting is pernicious, what might encourage individuals to vote across ethnic lines? According to the social capital literature, social trust is essential in building a robust democracy as it has been known to decrease discrimination and increase willingness to cooperate with others at the individual level, and improve collective action, economic growth and institutions at the national level\cite{uslanerMoralFoundationsTrust2002, bigelowDemocracyAmericaVolume1899, inglehartTrustWellbeingDemocracy1999, putnamWhatMakesDemocracy1993}. Social trust’s known ability to bridge individuals and groups, and thus promote democracy brings me to my research question; can an increase in trust across ethnic groups affect individuals’ voting behavior in contexts where ethnicity is salient?
\paragraph{}
I argue that the detrimental effect of ethnic diversity on democratic consolidation will be less prominent in contexts where individuals extend trust beyond their own ethnic groups (i.e., display high out-group trust). In particular, I look at the relationship between the radius of trust (i.e. the level of in- and out-group trust) and the extent of ethnic voting\cite{houleDoesEthnicVoting2018}. Since trust and voting behavior vary from individual to individual, I study this question at the individual level. The mechanisms through which the radius of trust determines the extent of ethnic voting are 1) a voter’s propensity to credit or discredit positive information on non-co-ethnic candidates or parties (information receptivity mechanism), and 2) a voter’s expectation of both co-ethnic and non-co-ethnic voter’s voting behavior (collective action mechanism).
\paragraph{}
This study on the role of out-group trust on democratic consolidation in ethnically diverse and salient settings speaks to a number of literatures. First, this study can add to the ethnic voting literature, particularly to the discussion on the conditions under which ethnicity is a significant predictor for vote choice\cite{chandraWhyEthnicParties2004, conroy-krutzInformationEthnicPolitics2013, posnerPoliticalSalienceCultural2004, dunningCrosscuttingCleavagesEthnic2010}. Second, it can contribute to the on-going debate in the social capital literature on whether and how social trust contributes to democratic development\cite{almondCivicCulturePolitical1989, inglehartRenaissancePoliticalCulture1988, mullerCivicCultureDemocracy1994a, inglehartPoliticalCultureDemocracy2003, rafaellaportaTrustLargeOrganizations1997, putnamWhatMakesDemocracy1993, riceSocialCapitalGovernment2001, knackSocialCapitalQuality2002a, uslanerMoralFoundationsTrust2002, backWhenTrustMatters2016, crepazWhatTrustGot2017a}. By considering the role of social trust in the relationship between ethnic identity and vote choice, we can not only learn about the extent to which ethnicity becomes the prime heuristic for people’s vote choice, but also whether social trust is a significant predictor of voting behavior in ethnically salient contexts.
\paragraph{}
The paper will proceed as follows. First, I explain why ethnic voting is important to consider when studying democratic consolidation and how it is detrimental to its progress. This is followed by a brief literature review on possible solutions for the negative consequences of ethnic diversity, including increasing social trust across ethnic lines. Then I describe explanations for why people vote along ethnic lines and discuss how out-group trust could possibly deter individuals from voting along ethnic lines. The following section lay out my proposed mechanisms for how out-group trust could lower the likelihood of individuals voting along ethnic lines. These are 1) the information receptivity mechanism, and 2) the collective action mechanism. This is then followed by a empirical and results section. 

\section{Literature Review}
\subsection{Ethnic Voting and Democratic Consolidation}
A democracy is consolidated, according to Linz and Stepan\cite{linzBreakdownDemocraticRegimes1978}, when democracy itself is “the only game in town.” When change is made, it is made through the democratic processes institutionalized in that country, rather than through authoritarian measures. Diamond\cite{diamondDemocraticConsolidation1994} says that democratic is consolidated when it “becomes so broadly and profoundly legitimate among its citizens that it is very unlikely to break down.” The possibility of a single person or party taking power is unlikely to happen, because the norms of democracy have become engrained in the system. A key component of both these characterizations of consolidation is elections. While consolidation also includes rule of law, independent judiciary, and a robust civil society, competitive elections are the base upon which these other factors build\cite{linzConsolidatedDemocracies1996}.
\paragraph{} 
Competitive elections guarantee a continuation of democracy because the contestation between candidates or parties prevents a single authority from staying in power indefinitely\cite{przeworskiDemocracyDevelopmentPolitical2000}. Ethnic voting or voting using ethnic cues to decide who to vote for, on the other hand, can be detrimental for democratic consolidation as it can undermine the competitive electoral process. According to Houle\cite{houleDoesEthnicVoting2018}, ethnic voting poses a danger to democratic consolidation for three reasons: ethnic voting 1) reduces ex ante uncertainty of voting, which is a fundamental characteristic of democracy\cite{przeworskiDemocracyDevelopmentPolitical2000}, 2) encourages patronage politics\cite{chandraWhyEthnicParties2004}, and 3) pushes candidates to take extreme policy stances leading to polarization\cite{horowitzEthnicGroupsConflict1985, rabushkaPoliticsPluralSocieties1972, chandraWhyEthnicParties2004, houleDoesEthnicVoting2018}.
\paragraph{}
The first point refers to the Przeworski’s definition of democracy, which he defines as a “system in which incumbents lose elections and leave office when the rules so dictate”\cite{przeworskiDemocracyDevelopmentPolitical2000} (54). The key characteristic of democracy he argues is contestation in the form of elections. For elections to be considered legitimate, they must fulfill three criterion: 1) ex-ante uncertainty (anyone can win), ex-post irreversibility (losers do not try to reverse results), and repeatability\cite{przeworskiDemocracyDevelopmentPolitical2000} (16). Ethnic voting makes it highly likely that the first criteria, ex ante uncertainty, will be violated. When politics are divided along ethnic lines, politicians are likely to appeal to their co-ethnic voters and those voters are more likely to vote for them. Since ethnicity is a sticky trait, voting along ethnic lines make the electoral outcomes more predictable. As ethnicity becomes more important to the voters, the demographics of the country will pre-determine who the winner and loser will be. An example of a country in which ethnicized politics has led to long term rule for particular parties is Kenya. Here, politics have always been dominated by parties led by Kikuyus , the biggest ethnic group in Kenya. On the other hand, in places where ethnicity is not the key factor for vote choice, the electoral outcomes will be more difficult to determine as voters may be more likely to switch parties based on their policies and past performances.
\paragraph{} 
Decreased unpredictability of electoral outcomes is bad for democracy as it undermines the legitimacy of the institution, which then discourages electoral losers from participating in future elections and having trust in their outcomes. Since they are likely to find the electoral results untrustworthy, they will have little interest in supporting the regime. Rather, they may have incentive to undermine democracy by staging a coup and installing a government led by their ethnic group or not partaking in electoral processes that further decreases their legitimacy. Groups in power, on the other hand, may try to consolidate their power by weakening the rule of law, taking away minority rights, or even by staging self-coups. All of these efforts from either side can lead to the fall of democracy.
\paragraph{} 
A second mechanism through which ethnic voting can erode democracy is by encouraging patronage politics. Patronage politics refers to a spoils system in which electoral winners exchange favors for votes. In places where votes are based on the candidate’s ethnicity, incumbents are less interested in the well-being of their citizens as a whole and more focused on pleasing their co-ethnic constituents. As a result, the incumbent is less likely to distribute public goods that benefit the country as a whole and more likely to give up patronage goods (e.g., provide public sector jobs) to their supporters. On the other hand, countries that do not vote along ethnic lines are more likely to eschew patronage politics and instead incumbents are likelier to appeal to all voters by providing public goods to the whole population.\cite{chandraWhyEthnicParties2004}
\paragraph{}
Patronage politics excludes electoral losers from accessing state resources, which in turn harms their well-being. Being excluded from accessing well-paid jobs can directly harm their socio-economic status, which in turn increases the economic inequality between the electoral winners and losers. The inequality then becomes a source of grievance leading to conflict that erodes and destabilizes democracy\cite{houleDoesEthnicVoting2018}.
\paragraph{}
Lastly, ethnic voting can harm democracy via ethnic out-bidding and resulting polarization. Ethnic out-bidding refers to the process where elites within the same group compete for votes by taking on a more extreme position than the other. When voting is primarily based on ethnicity, appealing to non-co-ethnic voters becomes unnecessary. As a result, candidates become more and more polarized in their stance as they try to outbid their competing co-ethnic candidate. The radicalized policies and rhetoric drive ethnic and co-ethnic groups further apart from one another, which can then lead to an emergence of “pernicious polarization”, a phenomenon where a society splits into mutually distrustful “us” versus “them” camps\cite{mccoyPolarizationGlobalCrisis2018}. In an extremely polarized environment, politicians are motivated to appeal to voters by proposing extreme policies, which favor co-ethnics and discriminate against non-co-ethnics. Voters, on the other hand, are influenced to loath, fear and distrust non-co-ethnics, which can in worst case scenarios lead to civil unrest and conflict\cite{bhavnaniEthnicPolarizationEthnic2009, devottaEthnicOutbiddingEthnic2005}.
\paragraph{}
Polarization along ethnic lines, according to McCoy and Somer\cite{mccoyTheoryPerniciousPolarization2019}, is especially detrimental to democracy as compared to cleavages based on issues or values. This is because cleavages formed around identity and belonging raises the “question (of) who has the right to live in a polity as a full citizen and whether one group can claim exclusive legitimacy to represent the citizens in the government”\cite{mccoyTheoryPerniciousPolarization2019} (263-64). Since these issues question the very existence of individuals, decisions derived from ethnic politics will directly affect the daily lives of the people. As electoral losers seek to regain power, they may work against the norms of competitive elections. Winners may also work against democratic norms in their attempts to hold onto power\cite{mckennaAreDiverseSocieties2018}. With violation of democratic norms from both sides, the country faces the danger of democratic backsliding. Such was the case in Kenya during the 2007 elections, where the incumbent’s alleged electoral manipulation led to the outbreak of ethnic violence targeting the incumbent’s ethnic group.
\paragraph{} 
Empirical studies support the theorized detrimental effect of ethnic voting on democracy. Results from Houle’s\cite{houleDoesEthnicVoting2018} study on ethnic voting and democracy across 58 democracies, as shown in Table 1, indicate a negative relationship between ethnic voting and democracy. To measure ethnic voting, Houle calculated the degree to which people of a given group vote for different parties than other groups of the same country. The score ranges from 0 to 1, where 0 indicates that members of group i vote in exactly the same way as other groups from the same country and 1 where members of group i vote strictly along ethnic lines. For levels of democracy, he used both the Polity score and the Freedom House score. The scores are based on a number of criteria including presence of a competitive election. The score for Polity ranges from -10 to 10, while the latter from one to seven, where higher scores indicate higher democracy or greater democratic consolidation. If countries score high on these measures, they are more likely to be democratically consolidated, including holding competitive elections. Using these measures, Houle finds ethnic voting is significantly correlated with a reduction in the quality of democracy. A country with a Polity score of 6.0, for example, would have a Polity score of 6.35 if its ethnic voting level (GVF)   was at the 5th percentile of the distribution. On the other hand, if the same country’s ethnic level were at the 95th percentile of the distribution, its Polity score would be 5.93. 
\subsection{Social Trust and Political Participation}
Given that ethnic voting appears to be detrimental to democratic consolidation via its impact on voting behavior, how can we discourage voting along ethnic lines? Social capital scholarship provides insight into how social trust can encourage people vote across ethnic lines.
\paragraph{} 
According to the social capital literature, social trust is a key foundation for democratic consolidation. Social trust is defined as a general disposition to what extent one trusts strangers or unfamiliar others\cite{uslanerSegregationMistrustDiversity2012, uslanerMoralFoundationsTrust2002}. It has been argued that social trust makes associations easier to create as it cuts down transaction costs related to formal coordination mechanisms like contracts, hierarchies, bureaucratic rules, and others\cite{fukuyamaTrustSocialVirtues2000, putnamBowlingAloneAmerica2000, uslanerMoralFoundationsTrust2002, warrenTrustDemocracy2018}. Elections is one area that can benefit from associations fostered by social trust. According to Keefer, Scartascini, and Vlaicu\cite{keeferSocialTrustElectoral2019}, low voter trust in each other is a fundamental concern when it comes to the quality of government. They argue that if voters do not trust that their fellow voters to act with them to hold politicians accountable, politicians will have less of a reason to fear the electoral consequences of breaking their promises\cite{keeferSocialTrustElectoral2019} (2). Using Latin American data, they find a strong correlation between low trust and preferences for policies associated with low quality and populist governments. As such evidence shows, social trust is closely associated with democratic consolidation as it has the ability to encourage collective action in keeping politicians accountable.
\paragraph{}
Empirical studies provide support for association between social trust and various aspects of democracy. At the country-level, studies find that social trust, usually measured as the percentage of respondents who agreed to the statement ‘most people can be trusted’, is a significant predictor for stable democracy, levels of democracy, and years of continuous democracy\cite{inglehartModernizationPostmodernizationCultural1997, inglehartTrustWellbeingDemocracy1999, mullerCivicCultureDemocracy1994b}. At the macro-level (e.g., national and community), studies find similar results where social trust is positively correlated with economic growth\cite{fukuyamaTrustSocialVirtues1995, knackDoesSocialCapital1997, dasguptaSocialCapitalMultifaceted2000}, lower crime rates\cite{jacobsDeathLifeGreat1992, wilsonTrulyDisadvantagedInner2012}, more responsive government\cite{putnamWhatMakesDemocracy1993}, and favorable view of the government\cite{riceSocialCapitalGovernment2001, knackSocialCapitalQuality2002, laportaQualityGovernment1999}. At the individual-level, high levels of social trust significantly predicts high levels of confidence in government (Brehm and Rahn 1997) and higher likelihood of protest\cite{bensonInterpersonalTrustMagnitude2004}.
\paragraph{}
On the other hand, absence of social trust has been found to have a detrimental effect on social and political stability. Previous studies on diverse societies, where the level of social trust is generally found to be low\cite{dinesenEthnicDiversitySocial2020}, find a positive association with conflict\cite{varshneyEthnicityEthnicConflict2009}, poor governance\cite{alesinaPublicGoodsEthnic1999}, low social capital\cite{alesinaEthnicDiversityEconomic2005}, and poor economic performance\cite{alesinaEthnicDiversityEconomic2005, easterlyAfricaGrowthTragedy1997}.
\paragraph{}
In contexts where ethnic identity is salient, social trust towards non-co-ethnics or out-groups matters. When social, political, and economic aspects of life are divided along ethnic lines, non-co-ethnic individuals or groups become potential competitors for resources. Since these non-co-ethnic members or groups are viewed as potential competitors, it is likely that the ability to trust these members and groups would also be low. When trust for non-co-ethnic or out-group members is low, cooperation across groups will be difficult, which then could have a detrimental effect on social and political stability. Studies examining the relationship between out-group trust and democracy find that countries with higher levels of out-group trust are more likely to score higher on the democratic scale. Delhey and co-authors\cite{delheyHowGeneralTrust2011} test the correlation between the radius of trust towards out-group members  and democratic awareness and level of democracy across 51 countries using the World Values Survey data. They find a significant and positive association between trust and two measures of democracy.
\paragraph{}  
At the individual-level, Crepaz and co-authors\cite{crepazWhatTrustGot2017} also use World Values Survey data to find that individuals with high levels of out-group trust  participate more actively in nonconventional political activity, such as participating in demonstrations, boycotts, and signing a petition. They also find that the presence of out-group trust had a slightly negative impact on voting. They explain that out-group trusters are more likely to engage in unconventional political behavior than conventional ones because they are “other regarding,” altruistic, and extroverted\cite{stollePoliticsSupermarketPolitical2005}. Their motivation for political participation lies not only in self-enrichment but also the pursuit of the common good. Out-group trusters, therefore, are more likely to engage in unconventional political activities that can demand change and solve collective problems. 
\subsection{Non-Co-Ethnic Trust and Ethnic Voting}
While previous studies on out-group trust and democracy show evidence of a positive relationship, scholarship has not addressed how trust across ethnic groups impacts ethnic voting and the mechanisms through which out-group trust influences voting behaviors. First, studies on out-group trust and democracy have aggregated multiple trust measures into a single index. Studies using the World Values Survey data construct an out-group trust index by averaging the level of trust across three groups: people you meet for the first time, people of another religion, and people of another nationality, none of which directly addresses ethnic lines. While the question of religion may capture trust across ethnic groups in some contexts, it may not be the case for countries where ethnicity and religious diversity do not overlap with one another. As a result, the question of trust towards people of another religion would not capture the ethnic tension the country is suffering from. By looking specifically at non-co-ethnic trust, one form of out-group trust, I hope to better understand the relationship between out-group trust and democracy.
\paragraph{}
Second, research on out-group trust and voting in democracies has yet to look at the effect of non-co-ethnic trust on ethnic voting. Previous studies have looked at the relationship between out-group trust and type of political participation individuals engage in, but they did not take into consideration ethnic contexts and how it would alter their voting behaviors. As a result, this paper seeks to understand the effects of trust across ethnic groups on individuals’ motivation to vote along ethnic line. If social trust has the ability to promote democratic behaviors, as per the social capital literature, we should expect to see a decreased motivation to vote along ethnic lines among those with high levels of non-co-ethnic trust. Furthermore, I seek to test the mechanisms through which non-co-ethnic trust has on voting behaviors in ethnically salient contexts. While the literature on social trust and political participation hint at a number of mechanisms through which trust influences voting behavior, it has yet to be explicitly tested for. 

\section{Theory on Outgroup Trust and Voting Behavior in New Democracies}
\subsection{Definitions and Concepts}
Trust refers to the belief that “others will not act opportunistically to take advantage of them”\cite{keeferSocialTrustElectoral2019}. Trust, according to the social capital literature, is considered beneficial for societies as it stimulates cooperation between citizens in general, including those that are divided socially and culturally\cite{bigelowDemocracyAmericaVolume1899, uslanerMoralFoundationsTrust2002}. Trust, especially trust extended to “strangers”, enhances feeling of common moral foundations, identity, and norms, all of which motivate people to achieve common goals that contribute to a democratic society\cite{putnamBowlingAloneAmerica2000, oskarssonGeneralizedTrustPolitical2010}.
\paragraph{}
In the context of ethnically divided societies, people’s ability to trust out-group members, seems like a possible solution to the deleterious consequences of voting along ethnic lines. Here, I theorize that voters who are able to trust individuals outside their identity group, as in voters with a larger radius of trust, are more likely to 1) incorporate information on competing candidates or parties into their voting decision (information receptivity mechanism) and 2) have confidence that non-co-ethnic voters will vote for qualified candidates or parties that will distribute public goods (collective action mechanism).
\paragraph{}
Before I explain the mechanisms on the relationship between radius of trust and voting behavior, I will define concepts relevant to the theory. First, radius of trust refers to the width of the “circle of people among whom cooperative norms are operative”\cite{fukuyamaTrustSocialVirtues2000}. When the radius of trust is narrow, trust is extended to people who are familiar to you including family members, neighbors, people you know or have met before, people of the same ethnicity, religion, age and so on. This type of trust is also known as particularized, in-group, specific, or “thick” trust\cite{zmerliSocialTrustAttitudes2008, delheyHowGeneralTrust2011}. A wide radius of trust, on the other hand, refers to trust in strangers and people whom we have little knowledge about. This type of trust is referred to as generalized, “impersonal” or “thin” trust\cite{gaidyteExplainingPoliticalParticipation2015}. According to Delhey, Newton and Welzel\cite{delheyHowGeneralTrust2011}, as the radius of trust increases so does the circle of cooperation. In contexts where ethnicity is a salient identity, “thick” trust is also referred to as in-group trust as in these settings friends and family members usually come from the same ethnic group. “Thin” trust, on the other hand, is often labeled as out-group trust or trust extended to those beyond one’s in-group members. In the next section, I present three possible mechanisms through which the radius of trust can influence the likelihood of voting along ethnic lines. 
\subsection{Why New Democracies?}
In this paper, I test whether the relationship between outgroup trust and voting behavior travels across new democracies. The reason for focusing on new democracies is because they may be more prone and vulnerable to ethnic voting. In new democracies, weak opposition parties’ inability to credibly promise to enact policies drives clientelism\cite{keeferClientelismCredibilityPolicy2007a}. This leads to politicians distributing goods to targeted groups rather than providing public goods. In such an environment, voters are likely to respond to such appeals by voting along ethnic lines.
\paragraph{}
Furthermore, citizens in new democracies, according to Letki\cite{letkiTrustNewlyDemocratic2018}, tend to rely on in-group trust compared to those in consolidated democracies. While not always the case, new democracies with an authoritarian past experiencing transition to democracy and market economy tend to have on average high levels of trust towards immediate friends and family but low levels of trust towards strangers. They are less likely to take risks that involve trusting strangers and may on occasions try to exploit the “other” in worry that others do not share their values\cite{banfieldCorruptionFeatureGovernmental1975, uslanerCivicEngagementParticularized2016}. When people withdraw from wider contact, they will not be able to reap the benefits of social capital. They, according to Uslaner and Conley\cite{uslanerCivicEngagementParticularized2016}, may “at best become hermits isolated from civic engagement. At worst they might reinforce prejudices against strangers when they interact only with people like themselves” (333). Social isolation resulting from low levels of out-group trust can further divide societies, reinforce prejudices, and in some instances lead to conflict that can lead to the destabilization of democracy.
\paragraph{}
In general, new democracies, compared to consolidated democracies, may be more prone to voting along ethnic lines and have lower levels of out-group trust. This, however, is not always the case; there is still variation in the levels of ethnic voting and out-group trust among new democracies. Research shows that ethnicity is not always the key predictor of voting behaviors\cite{horowitzEthnicitySwingVote2019, basedauEthnicityPartyPreference2011, houleStructureEthnicInequality2018} and that outgroup trust is not always low in all new democracies. According to Inglehart\cite{inglehartTrustWellbeingDemocracy1999a}, people living in countries with legacies of oppression are less likely to trust their fellow citizens nor participate in civic life.
\paragraph{}
In addition to variation in ethnic voting and outgroup trust levels, I have reason to believe social trust is a stronger predictor for voting behavior among new democracies than established democracies. Researching focusing on the differences between new and consolidated, historically, Western democracies find that, unlike effective and responsive political institutions present in consolidated democracies, states transitioning to democracies often suffer from institutional deficiencies early in their tenure\cite{huntingtonHowCountriesDemocratize1991, sorensenDemocracyDemocratizationProcesses2008}. In places where institutions are well-developed, as in the case of most consolidated democracies, people may be less reliant on social trust to navigate the world. Well-developed institution, free of corruption and discrimination, can help people live their day-to-day life without the fear of being cheated and taken advantage of. In countries with underdeveloped institutions, on the other hand, may need to rely instead on the help of their community, which includes strangers, to navigate their daily life. Furthermore, in consolidated democracies, it will be harder to tease apart the relationship between outgroup trust, institutional quality, and voting behavior as it is uncertain what factors enforce what. But considering the general low quality of institutions and out-group trust  among new democracies, it would be easier to tease apart the true effect of out-group trust on ethnic voting. As a result, considering the weak institutional strength, I argue that outgroup trust is going to be a strong predictor of voting behavior across new democracies.  

\subsection{Theory on Outgroup Trust and Voting Behavior}
In this section, I propose two mechanisms that explain the relationship between outgroup trust and voting behavior in ethnically salient contexts. They are the information receptivity mechanism, and collective action mechanism. 
\subsubsection{Mechanism 1: Information receptivity mechanism}
The first mechanism that explains the relationship between the radius of trust and the likelihood of voting along ethnic lines is the voter’s propensity to consider different types of candidate information. This cognitive explanation hypothesizes that the radius of trust determines whether a voter, when receiving information about co-ethnic and non-co-ethnic candidates or parties, credits or discredits that information.  This explanation adds to the information and accountability literature, which examines the effect of electoral information on voting behavior. As voters have access to additional information on politics, thereby cultivating a more-informed electorate, the salience of ethnic identity divisions in democratic politics will be reduced. The general argument here is that access to additional information on politics or cultivating an informed electorate may help reduce the salience of ethnic identity division in democratic politics.
\paragraph{}
While there is evidence that negative information on co-ethnic candidates or parties will dampen co-ethnic voter support\cite{conroy-krutzInformationEthnicPolitics2013}, there is also evidence that voters selectively choose the information they want to consider when making their vote choice\cite{adidaReducingReinforcingInGroup2017}. Contrary to the general expectation that increased information about candidate quality will reduce the importance of ethnicity in shaping one’s overall voting decision, Adida and co-authors instead find that voters engage in ethnically motivated reasoning where they consider positive information about co-ethnics as relevant and negative information as irrelevant to their vote choice. The opposite was true for non-co-ethnic members where voters considered positive information as irrelevant while negative information was relevant to their vote choice. In general, voters appear to choose what information they incorporate into their vote preferences based on their ethnic group membership.
\paragraph{}
I argue that increasing out-group trust can dampen people’s desire to engage in ethnically motivated reasoning and instead incorporate negative (positive) information on co-ethnic (non-co-ethnic) candidates or parties more seriously in their voting decisions. When trust is extended to those beyond their in-group members, people may deem non-co-ethnic individuals as trustworthy and honest (i.e., as someone who would not betray them). For individuals with high levels of out-group trust, positive information about non-co-ethnic candidates presents useful and believable information to consider when determining who to vote for, because the individual deems the non-co-ethnic members to be trustworthy and honest. As a result, these individuals, when provided both positive and negative information on co-ethnic and non-co-ethnic candidates, are more likely to consider all types of information when making their vote choice. With all the information they have on co-ethnic and non-co-ethnic candidates, they will be able to vote for a more qualified candidate with higher accuracy. As a result, voters with high out-group trust (i.e., voters with a wider radius of trust) are less likely to engage in ethnic voting than the low out-group trusting voters (i.e., voters with a narrower radius of trust), who are more likely to engage in ethnically motivated reasoning.
\paragraph{}
The information receptivity mechanism is somewhat overlapping with the network mechanism, but I believe the two are conceptually distinct. For example, it may be case where one has a fairly homogenous network, but they may still be willing to accept information from a non-co-ethnic they encounter due to pre-existing levels of out-group trust (information receptivity mechanism). Or it may be the case that one encounters more kinds of information because they have a heterogenous network and high levels of out-group trust (network mechanism). On the other hand, it could be the case that those with a diverse network can still be prejudiced against information coming from a non-co-ethnic due to low levels of out-group trust.   
\begin{figure}[H]
	\centering
	\includegraphics[scale=0.35]{Mech2A}
\end{figure}
\subsubsection{Mechanism 2: Collective action mechanism}
The second mechanism through which out-group trust can discourage voting along ethnic lines is through its effect on a voter’s perception that non-co-ethnic and co-ethnic voters will elect politicians that are qualified and distribute public goods. This mechanism differs from the two previous mechanisms because it is not a story of information but rather about individuals’ expectations about others’ voting behavior. This is similar to the ‘strategic selection mechanism’ theorized by Habyarimana and co-authors\cite{habyarimanaWhyDoesEthnic2007}, who argue that there are higher levels of public goods provision in ethnically homogenous communities because there exists a norm that cooperation among co-ethnics should be reciprocated and defections should be sanctioned. This theory assumes that in ethnically diverse societies, on the other hand, public goods provision is low as there is no unified norm of cooperation and sanctions.
\paragraph{}
The collective action mechanism proposed here frames individuals’ actions as also based on their expectations of others’ voting behavior, but not necessarily based on existing norms. I argue that individuals who extend trust towards non-co-ethnics are more likely to believe their out-group counterpart will cooperate and not defect in their voting decisions. Voters, when calculating their voting strategy, consider not only the competence of the candidates or parties, but also the strategy of fellow voters. They want their votes to contribute to the overall outcome and are likely to cast their vote for a candidate who is likely to win and likely to benefit the voter after elections as a result. When considering the strategic characteristic of voters, how individuals view others, and their intentions becomes crucial for one’s vote choice. According to Keefer, Scartascini, and Vlaicu\cite{keeferSocialTrustElectoral2019}, social trust is important in increasing the quality of government because it lowers the cost of collective action of demanding a better government. They argue that if voters can trust the other to contribute to the collective good of monitoring and expelling poorly performing incumbents, there is a higher incentive for individual voters to vote for qualified candidates or parties. Furthermore, the ethnic voting literature suggests negative evaluations of non-co-ethnics play an important role in motivating voting behavior. Across contexts, scholars have found evidence that prejudice and fears about the out-group plays a motivating role in co-ethnic voting\cite{kinderPrejudicePoliticsSymbolic1981, longDeterminantsEthnicVoting2012}. This is especially the case when there is social, political, and economic inequality across groups\cite{batesEthnicCompetitionModernization1974}. In such contexts, individuals who hold prejudicial views are less likely to support policies that benefit the out-group\cite{snidermanFallacyDemocraticElitism1991}. This complements the idea that ethnic voting tends to be prevalent in contexts where the other cannot be trusted, and thus that individuals will always vote in a way that disfavors the out-group rather than pursuing tactics that benefit the country as a whole.
\paragraph{}
Based on Keefer, Scartascini and Vlaicu\cite{keeferSocialTrustElectoral2019}, individuals with high levels of out-group trust are more likely to be optimistic about a non-co-ethnic voter’s openness to the idea of voting for a qualified candidate. High out-group trusters, compared to low out-group trusters, will have lower prejudice towards out-group members, and thus will tend not to think about politics from an “Us vs. Them” perspective. Rather, they are more likely to focus on what benefits not only their group, but the country as whole. In competitive electoral contexts, individuals, when thinking about their vote choice, are going to simultaneously think about who out-group members will vote for. Out-group trusters are more likely to think about others as allies rather than enemies, who may hold the same type of mindset as themselves. In this mindset, they may predict that non-co-ethnic voters are less likely to vote along ethnic lines and instead vote for a competent candidate or party. As a result, the individual is less likely to engage in ethnic voting than individuals with low levels of out-group trust. In other words, when voters are able to trust that non-co-ethnics will incur some costs of contributing to the collective good of keeping qualified candidates in power and expelling poor performing candidates, they are more likely to vote for the more qualified candidate, regardless of ethnicity.
\begin{figure}[H]
	\centering
	\includegraphics[scale=0.35]{Mech3A}
\end{figure}

\subsection{Hypotheses}
Based on the theoretical discussion above, I propose three hypotheses.\\
\\\textbf{H1.} Individuals with a \textit{high} level of out-group trust are \textit{less} likely to vote for a \textit{co-ethnic} candidate.\\
\\\textbf{H2.} Individuals with a \textit{high} level of out-group trust and \textit{high} propensity to consider different types of candidate information are \textit{less} likely to vote for a \textit{co-ethnic} candidate.\\
\\\textbf{H3.} Individuals with a \textit{high} level of out-group trust and \textit{high} level of information on the voting decision of in-group and out-group members are \textit{less} likely to vote for a \textit{co-ethnic} candidate.\\
\section{Research Design}
\subsection{Data}
To test the relationship between outgroup trust and voting behavior among in individuals in new democracies, I used data from the Afrobarometer Survey and Global Leadership Project. I chose to examine new democracies in Africa because these countries are not only ethnically diverse but ethnicity is also a salient identity when it comes to political mobilization. Here, I employ the definition of new democracies used by Grewal and Voeten\cite{grewalAreNewDemocracies2015}, which includes countries that have a Polity IV score of 6 or higher for less than 30 consecutive years. There are 10 new democracies in Wave 3 (2005) of the Afrobarometer survey that fit this defintion in Africa\footnote{New democracies included in Wave 3 of the Afrobarometer survey are Benin, Botswana, Ghana, Kenya, Madagascar, Malawi, Mali, Namibia, Senegal, and South Africa} at 2005. I based this study on data from Afrobarometer Wave 3 as this survey wave included questions on outgroup/interethnic trust and voting behavior at the individual-level.
\paragraph{}
The Global Leadership Project (GLP) is a dataset that offers biographical information on leaders throughout the world, including members of the executive, the legislature, the judiciary, and other elites who hold informal power\cite{gerringWhoRulesWorld2019}. This dataset includes information on the ethnicity of leaders, which will be used as part of measuring ethnic voting. 

\subsubsection{Dependent Variable}
The main outcome variable is ethnic voting at the individual level. Here, ethnic voting is a dichotomous variable, operationalized as whether an individual voted for a co-ethnic candidate or not. To measure ethnic voting, the ethnicity of presidential candidates was matched with that of the respondent's ethnicity. Based on Afrobarometer's "If a presidential election were held tomorrow, which party's candidate would you vote for?" question, I made a list of presidential candidates or party leaders representing these parties of choice. Then the ethnicity of each party's presidential candidate or leader were identified using the GLP dataset. For those whose information was not available in the GLP dataset, I either located their ethnicity through a web search or left it blank. The errors that can arise from this coding will be discussed later in the limitations section. Once the ethnicity of presidential candidates or party leaders were identified, I matched their ethnicity with that of the respondent's ethnicity, as provided by the Afrobrometer. Respondents who voted for co-ethnic candidates were coded as 1 and those that voted for a non-co-ethnic candidate as 0. 

\subsubsection{Independent Variables}
The main independent variable is out-group trust. Out-group trust is operationalized as the level of trust an individual has towards a non-co-ethnic member. This is measured using responses to the Afrobarometer question of, "How much do you trust each of the following types of people? Kenyans from other ethnic groups." This is an ordinal variable, in which the responses range from 0(Not at all) to 3(I trust them a lot).
\paragraph{}
To account for the information receptivity mechanism, I included a variable counting the number of media sources respondents used to get their news. The Afrobarometer includes questions asking respondents, "How often do you get news from the following sources? Radio; Television; Newspaper." The responses range from 0(Never) to 4(Everyday), and I compiled the three questions on these sources into one measure by adding the responses together, where lower values indicate low media access and high values greater media access. I make the assumption that respondents with higher levels of out-group trust are more open to a wider variety of political information and thus will seek it from a variety of news sources. This question is not a direct measure of exposure to diverse information, which means results must be interpreted with caution. Using this measure, I examined the interaction effect of out-group trust and information diversity, and how this affects voting behavior.
\paragraph{}
To test the collective action mechanism, a measure on individual's frequency of political discussion with friends and family was included. The Afrobarometer includes a question asking, "When you get together with your friends or family, would you say you discuss political matters?" The responses range from 0(Never) to 2(Frequently). Here, I assume that those who discuss political matters frequently are more likely to share their vote choice with friends and family than those who do not. As a result, those who actively discuss political matters are more likely to be knowledgeable about vote choice of others, which in turn will influence their own vote choice. Again, this is not a direct measure for one's knowledege on the vote choice of others, therefore, results based on this measure must be interpreted with caution. Using this meausre, I looked at the interaction effect of out-group trust and active political discussion, and how this influenced the respondent's voting behavior. 

\subsection{Empirical Strategy}
A multi-level model was used to examine the relationship between out-group trust and ethnic voting as I believe voting behavior at the individual-level are influenced by individual, regional, and country level characteristics. Here it is assumed the level-1 observations are nested within level-2 units. To control for these characteristics, I include control variables at the individual(level-1) and country(level-2) level. At the individual-level, I control for five factors: age, gender, education, economic status, and political trust. I included political trust into the model as previous studies find a significant relationship between out-group trust and confidence in institutions\cite{caoIngroupOutgroupForms2015}, and I suspect political trust to have an effect on voting behavior as people with greater confidence in the institution are less likely to rely on informal cues such as ethnicity when making their vote choice.
\paragraph{}
At the country-level, I control for the three factors: GDP, years of democracy, and ethnic voting at the country level. I use Huber's\cite{huberMeasuringEthnicVoting2012} Group Voting Fractionalization measure, which measures the electoral distance between any two groups. The measure ranges from 0 to 1, where 1 refers to distance between the two groups, where all of \textit{i}'s supporters are from one group and all of \textit{j}'s supporters from a different group. Figures on GDP and years of democracy for 2005 were obtained from the World Bank and PolityIV dataset. This variable is included as I suspect out-group trust to be low in countries where politics is severely divided along ethnic lines and also voting behavior to be influenced by the voting pattern within the country.

\section{Results}
\subsection{Summary Statistics}
	\begin{table}[H]
	\setlength{\arrayrulewidth}{1mm}
	\setlength{\tabcolsep}{18pt}
	\renewcommand{\arraystretch}{2}
	\Huge
	\centering
	\resizebox{\textwidth}{!}{%
		\begin{tabular}{|c|c|c|c|c|c|c|c|c|c|c|c|c|}
			\hline
			Variable & n & mean & sd & median & trimmed & mad & min & max & range & skew & kurtosis & se\\
			\hline
			\multicolumn{13}{|c|}{Individual Level Variables}\\
			\hline
			Ethnic Voting & 9505 & 0.33 & 0.47 & 0 & 0.29 & 0.00 & 0 & 1 & 1 & 0.73 & -1.47 & 0.00\\
			\hline
			Out-group Trust & 9338 & 1.43 & 1.03 & 1 & 1.41 & 1.48 & 0 & 3 & 3 & 0.13 & -1.13 & 0.01\\
			\hline
			Age & 9415 & 37.46 & 14.93 & 35 & 35.87 & 14.83 & 18 & 115 & 97 & 0.89 & 0.30 & 0.15\\
			\hline
			Gender & 9505 & 1.48 & 0.50 & 1 & 1.48 & 0.00 & 1 & 2 & 1 & 0.07 & -2.00 & 0.01 \\
			\hline
			Education & 9476 & 3.01 & 2.02 & 3 & 2.95 & 1.48 & 0 & 9 & 9 & 0.19 & -0.41 & 0.02\\
			\hline
			Economic Status & 9462 & 2.11 & 1.40 & 2 & 2.14 & 1.48 & 0 & 4 & 4 & 0.05 & -1.28 & 0.01 \\
			\hline
			Political Trust & 8881 & 1.87 & 1.07 & 2 & 1.97 & 1.48 & 0 & 3 & 3 & -0.47 & -1.06 & 0.01\\
			\hline
			\multicolumn{13}{|c|}{Country Level Variables}\\
			\hline
			GDP & 9505 & 2269.64 & 2215.20 & 822.46 & 2111.72 & 745.63 & 289.56 & 5513.33 & 5223.77  & 0.56 & -1.54 & 22.72\\
			\hline
			Democratic Yrs & 9505 & 11.89 & 10.60 & 11.00 & 9.84 & 5.93 & 1.00 & 39.00 & 38.00 & 1.63 & 1.95 & 0.11\\
			\hline
			Country Ethnic Voting & 9505 & 0.18 & 0.08 & 0.16 & 0.17 & 0.07 & 0.06 & 0.33 & 0.27 & 0.58 & -1.32 & 0.03\\
			\hline	
	\end{tabular}%
	}
	\caption{Summary Statistics for Individual and Country Level Variables }
\end{table}
\paragraph{}
Table 1 reports the summary statistics for both individual and country-level variables included in the model. At the individual-level, there were a total of 9505 respondents who said they would vote for a party if the presential election were held tomorrow. For \textit{ethnic voting}, the main outcome variable, the responses were dichotomous where 0 was assigned to those who voted for a non-co-ethnic candidate and 1 for those who voted for a co-ethnic candidate. The average response score was 0.33, meaning more respondents voted for a non-co-ethnic candidate than a co-ethnic one. Regarding \textit{out-group trust}, the responses ranged from 0 to 3, where 0 signified no trust and 3 a lot of trust. The mean score was 1.43 meaning that the majority of respondents had very little trust towards non-co-ethnic members. The respondents were on average in their \textit{mid-thirties}, \textit{males}, \textit{received some primary schooling}, and \textit{had gone without cash income once or twice in the past year}. Lastly, respondents' level of \textit{political trust} was on the higher end, where the majority said they somewhat trusted their country's Parliament/National Assembly.
\paragraph{}
Ten Sub-Saharan new democracies varied in their level of GDP per capita, democratic years, and level of ethnic voting. For levels of GDP per capita, Malawi had the lowest level of GDP and Botswana the highest. For democracy years, I subtracted the year a country's polity score changed to a 6 from 2005. Mali was youngest democracy with a year of 1 and Botswana the oldest with a year of 39. Lastly, countries also varied in their country's ethnic voting score, with Senegal with the lowest level of ethnic voting, 0.059 and Kenya with the highest level 0.332.

\subsection{Results from Multilevel Analysis}
\begin{table}[H]
	\centering
	\includegraphics[scale=0.5]{maintrust}
	\caption{Results for Out-Group Trust and Ethnic Voting}
\end{table}
\paragraph{}
Table 2 presents the results of the multilevel analysis. Overall, the results show strong support for the first hypothesis and weak support for the second hypothesis, but find no support for the third hypothesis.
\paragraph{}
Comparing across model 1, 2, and 3, I find that the variable measuring out-group trust (\textit{ietrust}) remains significant even after including individual and country-level covariates. The coefficients across the model are negative and statistically significant meaning that my first hypothesis is supported. According to these models, individuals with a \textit{high} level of out-group trust are less likely to vote for a \textit{co-ethnic} candidate, and individuals with a \textit{low} level of out-group trust are more likely to vote for a \textit{co-ethnic} candidate. Furthermore, using the ANOVA test (table 3), I compare the deviance statistics of model 3 \textit{without}(Model 3a) and \textit{with}(Model 3b) the \textit{ietrust} variable. The deviance statistics decreases and is significant after including the \textit{ietrust} variable, indicating that this is a better fitting model.
	\begin{table}[H]
	\setlength{\arrayrulewidth}{1mm}
	\setlength{\tabcolsep}{18pt}
	\renewcommand{\arraystretch}{2}
	\Huge
	\centering
	\resizebox{\textwidth}{!}{%
\begin{tabular}{|c|c|c|c|c|c|c|c|c|}
	\hline
	Model & npar & AIC & BIC & logLik & deviance & Chisq & Df & Pr(>Chisq) \\
	\hline
	Model 3a & 10 & 10194 & 10265 & -5087.1 & 10174 &  & & \\
	\hline
	Model 3b & 11 & 10187 & 10265 & -5082.5 & 10165 & 9.2457 & 1 & 0.002361 ***\\
	\hline
	\multicolumn{9}{|c|}{Note: *p$<$ 0.1; **p$<$0.05; ***p$<$0.001}\\
	\hline
\end{tabular}%
}
\caption{ANOVA Test Comparing Models With and Without Out-Group Trust}
\end{table}
\paragraph{}
Here, I also find age, gender, education, and political trust variables to have a significant effect on voting behvior across individuals. Those who are \textit{older} and are \textit{females} tend to be more likely to vote for a co-ethnic candidate. Moreover, individuals with fewer years of education tend to also vote for a co-ethnic candidate. Interestingly, economic status had no impact on whether an individual votes along ethnic lines or not. Individual's trust towards Parliament or National Assembly also had an effect on the level of ethnic voting. Those with higher levels of political trust were more likely to vote for a co-ethnic candidate than those with lower levels of political trust. None of the country-level variables, however, are significant. I suspect that this is due to the small number of countries included in the model. 
\paragraph{}
Model 4 tests the second hypothesis, which states that individuals with a \textit{high} level of out-group trust and \textit{high} propensity to consider different types of candidate information are more likely to vote for a \textit{non-co-ethnic} candidate. By interacting variables \textit{ietrust} and \textit{information acc}, I test to see whether individuals with high levels of out-group trust \textit{and} greater exposure to information through media are less likely to vote along ethnic lines. Results show that the interactive term \textit{ietrust:information acc} is positive and significant. However, model coefficients are harder to interpret at face value for interactive terms. As a result, I examine how different levels of out-group trust has different effects for different levels of information access on ethnic voting.
\begin{figure}[H]
	\centering
	\includegraphics[scale=0.65]{PPEV1 copy}
	\caption{Predicted probabilities for Ethnic Voting with Interactive Term (ietrust*information acc) }
\end{figure}
\paragraph{}
Figure 1 is a visualization of the changes in outcome (ethnic voting) with changes in the main independent variable (out-group trust) at different levels of information access in a model with an interaction term. We can see from the figure that each regression line has different slopes for different level of media access. The green line indicates the effect of out-group trust on ethnic voting among individuals who had the highest level of access to information via media, the blue line with a medium level of information access, and green line for individuals with lowest level of information access. We can see that the red line has the steepest slope where as the green line has the flattest slope. This indicates that out-group trust tends to have the greatest effect on ethnic voting among those with a lowest level of information acess, followed by those with medium level access (blue), and the highest access (green). The results provide mixed support for my hypothesis as those with medium (blue) and low (red) levels of information access (blue) tend to behave in the hypothesized direction, but for those with high levels of information access out-group trust has neglible effect on ethnic voting. For those with the highest level of information access (green), there was no difference in the level of ethnic voting among those with low and high levels of out-group trust. Rather, it was the individuals with low levels of information access that showed the most difference in the levels of ethnic voting. A probably explanation for this result is that too much information can have a backfiring effect on people's vote choices. It may be that too much information confuses the voters and in turn they resort to basic cues such as ethnicity instead of incorporating information collected through media into their vote choice. 
\paragraph{}
Next, model 5 finds no support for the third hypothesis, which states that individuals with a \textit{high} level of out-group trust and \textit{high} level of information on the voting decision of in-group and out-group members are more likely to vote for a \textit{non-co-ethnic} candidate. When I include the \textit{pol discuss} variable into the model, \textit{ietrust} is no longer significant. Rather, I find evidence for the \textit{pol discuss} variable where individuals with greater level of political discussion with friends and family tend to vote for a co-ethnic. I suspect that either the variable is not accurately capturing people's experiences of having political discussion with not only co-ethnic but also non-co-ethnic members or that people in Sub-Saharan Africa rarely have political discussions with non-co-ethnic members. 

\section{Discussion and Conclusion}
In conclusion, the multilevel analysis finds a significant relationship between out-group trust and ethnic voting at the individual level. I, however, find limited support for the information receptivity mechanism where out-group trust seemed to have the greatest effect on ethnic voting among those with low exposure to information followed by medium level of information. Contrary to my theory, out-group trust seemed to have negligible effect among those with the highest exposure to information. Lastly, I find no significant support for the collective action mechanism. 
\paragraph{}
This study is suffers from a number of limitations. First, the main outcome variable, ethnic voting, has two major drawbacks.  First, party leaders may have more than one ethnicity to which they appeal to\cite{adidaSpousalBumpCrossEthnic2016}. As a result, my simplified method of matching the candidate's primary ethnicity to that of the respondents would result to Type II error (false negative). Voters could have responded to the candidates' secondary or spousal's ethnic appeal, but I could have miscoded it as non-ethnic voting because of my focus on the candidates' primary ethnicity. The second issue with this measurement is that not all parties mobilize constituents along ethnic lines. While some parties may mobilize their constituents according to their ethnic identity, others may appeal to their voters using other issues (e.g., income groups, ideology, policy, etc.). By assuming all parties as ethnic, I could have committed a Type I error (false positive), where I miscode respondent's voting decision as ethnic voting in cases where the non-ethnic party's ethnicity and the respondent's ethnicity happened to match. 
\paragraph{}
Another potential limitation of this study is the number of countries included in the study. Since there are only ten observations in level-2, it is difficult to make significant comparisons across these countries. Usually, it is recommended a multi-level model have at least 20 observations at each level to detect cross-level interactions\cite{kreftIntroducingMultilevelModeling1998}. This, however, was not possible for this round of analysis as I wanted to use the Wave 3 Afrobarometer survey because it included questions that closely measured my independent variable, out-group trust. In the more recent waves of the Afrobarometer, they include the question, which asks "For each of the following types of people, please tell me whether you would like having people from this group as neighbors, dislike it, or not care." While this does get at out-group trust somewhat, it is not a direct measure. 
\paragraph{}
This study makes some significant contribution to the literature on social capital and ethnic voting. While past studies have theorized and empirical proven social capital's effect on political participation, it has yet to make direct connections to voting in ethnically salient contexts. Results show that trust across ethnic groups has a significant effect on voting in places where politics is influenced by ethnic identity. Furthermore, this study also speaks to the ethnic voting literature as trust across ethnic groups can explain for some variation in the level of ethnic voting across individuals and countries. This shows that trust across ethnic group is taken into consideration when deciding who to vote for in the presidential election across Sub-Saharan countries. 
\paragraph{}
To better understand the relationship between out-group trust and ethnic voting and its mechanisms, future research must do the following. First, ethnic voting at the individual must be measured more acurrately. As aforementioned, I should be able to distinguish between ethnic and non-ethnic parties, and better determine which ethnic groups the ethnic parties are mobilizing. Second, to expand the number of level-2 observations, I should be determine whether the question on having non-co-ethnics as neighbors is a good proxy for out-group trust or not. If this turns out to be a good proxy then recent waves of the Afrobarometer survey and the World Values survey data can be used to furhter determine the relationship between out-group trust and ethnic voting. Third, to better account for the mechanisms, I should come up with better measures for information receptivity and knowledge on the voting intentions of others. To measure whether an individual is being exposed to a wider variety of information on both co-ethnic and non-co-ethnic candidates, I should be able to determine whether the media sources they access are biased or not. It could be possible that one is accessing media channels that are biased towards one's in-group. In this case, they could be exposed to more information but it would be heavily biased, which may motivate one to vote along ethnic lines. To better account for the collective action mechanism, I need to find a better measure for one's knowledge of not only their in-group member's voting intentions but also that of the out-group members'.  

\pagebreak
\section{Appendix}
\subsection{Variable Measurement and Data Source}
Afrobarometer Wave 3 (2005) 
\begin{itemize}
	\item Voting: Q99 If a presidential election were held tomorrow, which party's candidate would you vote for?
	\item Out-group trust: Q84D How much do you trust each of the following types of people: [Ghanaian/Kenyan/etc.] from other ethnic groups? (0=Not at all, 1=Just a little bit, 2=Somewhat, 3=A lot)
	\item Age: Q1 How old are you? 
	\item Gender: Q101 Respondent's gender (1=Male, 2=Female)
	\item Education: Q90 What is the highest level of education you have completed? (0= No formal schooling, 1= Informal schooling (including Koranic schooling), 2=Some primary schooling, 3=Primary school completed, 4=Some secondary school/ High school, 5=Secondary school completed/High school, 6=Post-secondary qualifications, other than university e.g. a diploma or degree from a technical/polytechnic/college, 7=Some university, 8=University completed, 9=Post-graduate)
	\item Economic status: Q8E Over the past year, how often, if ever, have you or anyone in your family gone without: A cash income? (0=Never, 1=Just once or twice, 2=Several times, 3=Many times, 4=Always)
	\item Political trust: Q55B How much do you trust each of the following, or haven't you heard enough about them to say: The Parliament/National Assembly? (0=Not at all, 1=Just a little bit, 2=Somewhat, 3=A lot)
	\item Information access: Q15(A-C) How often do you get news from the following sources? Radio; Television; Newspaper (0=Never, 1=Less than once a month, 2=A few times a month, 3=A few times a week, 4=Every day). I added the responses from the three questions. 
	\item Political discussion: Q17 When you get together with your friends or family, would you say you discuss political matters? (0=Never, 1=Occasionally, 2=Frequently)
	\item Majority: Q79 What is your tribe? You know, your ethnic or cultural group. 
	\item In-group attachment: Q84C How much do you trust each of the following types of people: People from your own ethnic group? 
	\item Proximity to the capital: square root of respondent's region of residence to the country's capital.  
\end{itemize}
World Bank
\begin{itemize}
	\item GDP per capita: log value of GDP per capita in year 2005. 
\end{itemize}
Polity IV dataset 
\begin{itemize}
	\item Years of democracy: log value of years since the country turned from a polity score of below 6 to a 6 or above 6. 
\end{itemize}
Huber's\cite{huberMeasuringEthnicVoting2012} ethnic voting dataset
\begin{itemize}
	\item Group Voting Fractionalization: log of value, which measures the electoral distance between any two groups. The measure ranges from 0 to 1, where 1 refers to distance between the two groups, where all of \textit{i}'s supporters are from one group and all of \textit{j}'s supporters from a different group.
\end{itemize}

\subsection{Additional Models}
\begin{figure}[H]
	\centering
	\includegraphics[scale=0.55]{additionalmodel}
\end{figure}

\bibliographystyle{plain}
\bibliography{Diss}

\end{document}

